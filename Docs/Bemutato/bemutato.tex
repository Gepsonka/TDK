\documentclass{article}
\usepackage[utf8]{inputenc}
\usepackage{hyperref}

\title{UAV security}
\author{Molnár Botond}
\date{December 2022}

\begin{document}

\maketitle

\section{Motiváció}

Kutatás motivációja egy LoRa-kommunikáción alapuló UAV felderítő rendszer implementálása, amely automatizált segítséget képes nyújtani a parti és határőrség, a szárazföldi erők és a polgári biztonsági személyzet számára a felderítésben és a fenyegetések észlelésében.

\section{A kutatás részletei}

\begin{flushleft}
A kutatás elsősorban a LoRa fizikai rétegén keresztül történő biztonságos kommunikáció létrehozására összpontosít. A titkosítási algoritmus kiválasztása, figyelembe véve a mikrokontrollerek által biztosított korlátozott erőforrásokat, a kulcscsere, Átfogalmaznám mert így inkább szószaporításnak tűnik laikusnak közben pedig nem az.
az ideális titkosítási algoritmus kiválasztása  és implementálása az adatok hatékony titkosításának és visszafejtésének érdekében.
Teendők közé tartozik még az új funciók tesztelése és elemzése éles környezetben

A LoRa technológia ideális csatorna a kommunikáció megvalósításához, mert viszonylag nagy hatótávolságú, megbízható adatátviteli tulajdonságokkal rendelkezik, a jelet nehéz zavarni és a hatótávolságához képest viszonylag sok adatot tud továbbítani. Végül pedig nagyfokú hibatűréssel rendelkezik.

E tulajdonságok lehetővé teszik a hatékony titkosított adattovábbítást.
\end{flushleft}

\section{Relevancia}

\begin{flushleft}
Napjainkban a katonai és a polgári szektorban is fokozatosan nő a felhasznált könnyű UAV-ok száma, például az Amazon csomagszállító drónjai vagy az orosz-ukrán háborúban mindkét fél által használt drónok formájában. Az eszközök számának növekedésével egyre nagyobb szükség lesz biztosítani az adatok és a platformok informatikai védelmét, mivel egyre gyakoribb szereplők lesznek mindennapjainkban. A jelenlegi állapot szerint a kiforratlan biztonsági infrastruktúra pedig súlyos támadásokhoz vezethet, amelyek nemcsak egy személy, hanem egy állam digitális tulajdonait is veszélyeztethetik.
\end{flushleft}

Biztonság

Az UAV biztonsága összetett feladat, figyelmet kell fordítani a hardveres védelemre, a tárolók titkosítására és a hálózat védelmére.

A tárolók titkosítása komoly biztonsági szempont, hiszen ha az eszköz elkapják, az adatok illetéktelen kezek számára hozzáférhetetlenek, a szoftver pedig nem fordítottan fejleszthető, így a rendszer érzékeny adatai nem szivárognak ki.

A tárolók titkosításának általános gyakorlata az AES titkosítás használata. Így csak az arra jogosultak férhetnek hozzá a tárolón lévő adatokhoz.

A hardveres védelemmel az eszköz védett a zavarás, a GPS hamisítás és az EMP ellen, amely károsítja az eszközt, elérhetetlenné teszi az eszközt az irányítóközpont számára, vagy hibás koordinátadatokat juttat az eszközbe és az érzékelőibe.

E fenyegetések ellen elsősorban úgy lehet védekezni, hogy az elektromos alkatrészeket interferencia- és elektromágneses impulzusok elleni eszközökkel védik, az antennákat elrejtik, és csalóantennákat alkalmaznak. A GPS-jel sértetlenségének biztosítása érdekében az amerikai hadsereg kifejlesztett egy titkosított módszert, amellyel titkosított GPS-koordinátákat szolgáltat platformjainak.

Vannak más módszerek is a GPS hamisítás észlelésére, amelyek gépi tanulási algoritmusokra és hardveres észlelésre támaszkodnak.

A rendszer kulcsfontosságú összetevőinek, például a giroszkópnak, a magnetométernek és a sebességszabályozónak is nagyfokú rendelkezésre állást kell biztosítani, mivel ezek kulcsszerepet játszanak a készülék levegőben tartásában. Ezeket az eszközöket jól le kell árnyékolni az elektromágneses interferenciával szemben.




Translated with www.DeepL.com/Translator (free version)

\section{Források}

\begin{flushleft}

\url{https://lirias.kuleuven.be/retrieve/460994}

