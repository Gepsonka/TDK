\documentclass{article}
\usepackage[utf8]{inputenc}
\usepackage{hyperref}

\title{UAV security}
\author{Molnár Botond}
\date{December 2022}

\begin{document}

\maketitle

\section{Motiváció}

Kutatás motivációja egy LoRa-kommunikáción alapuló UAV felderítő rendszer implementálása, amely automatizált segítséget képes nyújtani a parti és határőrség, a szárazföldi erők és a polgári biztonsági személyzet számára a felderítésben és a fenyegetések észlelésében.

\section{A kutatás részletei}

\begin{flushleft}
A kutatás elsősorban a LoRa fizikai rétegén keresztül történő biztonságos kommunikáció létrehozására összpontosít. A titkosítási technológia kiválasztása, figyelembe véve a mikrokontrollerek által biztosított korlátozott erőforrásokat, a kulcscsere, az adatok titkosításának és visszafejtésének megvalósítása, a fizikai biztoság vizsgálata.
Teendők közé tartozik még az új funciók tesztelése és elemzése éles környezetben

A LoRa technológia ideális csatorna a kommunikáció megvalósításához, mert viszonylag nagy hatótávolságú, megbízható adatátviteli tulajdonságokkal rendelkezik, a jelet nehéz zavarni és a hatótávolságához képest viszonylag sok adatot tud továbbítani. Végül pedig nagyfokú hibatűréssel rendelkezik.

E tulajdonságok lehetővé teszik a hatékony titkosított adattovábbítást.
\end{flushleft}

\section{Relevancia}

\begin{flushleft}
Napjainkban a katonai és a polgári szektorban is fokozatosan nő a felhasznált könnyű UAV-k száma, például az Amazon csomagszállító drónjai vagy a szomszédos invázióban mindkét fél által használt drónok formájában. Az eszközök számának növekedésével egyre nagyobb szükség lesz biztosítani az adatok és a platformok informatikai védelmét, mivel egyre gyakoribb szereplők lesznek mindennapjainkban. A jelenlegi állapot szerint a kiforratlan biztonsági infrastruktúra pedig súlyos támadásokhoz vezethet, amelyek nemcsak egy személy, hanem egy állam digitális tulajdonait is veszélyeztethetik.
\end{flushleft}

\section{Források}

\begin{flushleft}

Contemporary Macedonian Defence, 2021, June

\href{https://d1wqtxts1xzle7.cloudfront.net/72008152/Sovremena_Makedonska_Odbrana_br.40_en_2_-libre.pdf?1633795839=&response-content-disposition=inline%3B+filename%3DGREAT_BRITAIN_S_EXIT_FROM_THE_EU_BREXIT.pdf&Expires=1670532548&Signature=XdeR1thXJwZKficBvz4kdmL293VKzFvY7FDccbf65f8nLW2bOAqNZ7-qkc0y2wA1x6iE3LW9xF-xhDdhLu1lPU-B2STv62O4-KhOR5iXWdxrWibulCv4Mgy-jNi58CfCPbQ9VLtCZjW6XKu04kPXjzSpQAg61Ka8GzEJCOYe8We3GURPW9VHWiUXDaoelwsPeUEFCT00QS3q5fRA6wCheY~0JGEZR5vaY3qmlv9EMOoWE5O44Q88phDsSMw2fRlI~7fot0y4OH286eKjDtUPzUk6swKnSfAVV2HFWrLWqVv14fgZzGmpxN~llqoTxrOkmhDLCRjkQsoLDqs6Mu3UpA__&Key-Pair-Id=APKAJLOHF5GGSLRBV4ZA#page=118}{link}

\bigskip

Vikas Hassija, Vinay Chamola, Adhar Agrawal, Adit Goyal, Nguyen Cong Luong, Dusit Niyato, Fei Richard Yu, Mohsen Guizani, Fast, Reliable, and Secure Drone Communication: A Comprehensive Survey, 2021

\href{https://ieeexplore.ieee.org/abstract/document/9488323/authors#authors}{link}

\bigskip

Drone-Assisted Public Safety Networks: The Security Aspect, 2017, April

\href{https://link.springer.com/chapter/10.1007/978-981-15-5827-6_33}{link}

\bigskip

V. Porkodi, Saatvik Awasthi, Balamurugan Balusamy, Artificial Intelligence Supervised Swarm UAVs for Reconnaissance, 2019

\href{https://link.springer.com/chapter/10.1007/978-981-15-5827-6_33}{link}

\bigskip

Saeed Ullah Jan, Habib Ullah Khan, Identity and Aggregate Signature-Based Authentication Protocol for IoD Deployment Military Drone, 2021, September

\href{https://ieeexplore.ieee.org/abstract/document/9530542}{link}

\bigskip


Christian Bunse \& Sebastian Plotz, Security Analysis of Drone Communication Protocols, 2018, June

\href{https://link.springer.com/chapter/10.1007/978-3-319-94496-8_7}{link}

\bigskip

Seung-hyun Seo, 
Jongho Won, E. Bertino, Yousung Kang, Dooho Choi, A Security Framework for a Drone Delivery Service, 2016, June

\href{https://dl.acm.org/doi/abs/10.1145/2935620.2935629}{link}

\bigskip

Antonio Caruso, Stefano Chessa, Soledad Escolar, Jesús Barba, Juan Carlos López, Collection of Data With Drones in Precision Agriculture: Analytical Model and LoRa Case Study, 2021, april

\href{https://ieeexplore.ieee.org/abstract/document/9416287}{link}

\bigskip

Vageesh Anand Dambal, Sameer Mohadikar, Abhaykumar Kumbhar, Ismail Guvenc, Improving LoRa Signal Coverage in Urban and Sub-Urban Environments with UAVs, 2019, March

\href{https://ieeexplore.ieee.org/abstract/document/8730598}{link}

\bigskip

\end{flushleft}

\end{document}
